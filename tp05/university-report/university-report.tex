\documentclass[french,]{article}
\usepackage{lmodern}
\usepackage{amssymb,amsmath}
\usepackage{ifxetex,ifluatex}
\usepackage{fixltx2e} % provides \textsubscript
\ifnum 0\ifxetex 1\fi\ifluatex 1\fi=0 % if pdftex
  \usepackage[T1]{fontenc}
  \usepackage[utf8]{inputenc}
\else % if luatex or xelatex
  \ifxetex
    \usepackage{mathspec}
  \else
    \usepackage{fontspec}
  \fi
  \defaultfontfeatures{Ligatures=TeX,Scale=MatchLowercase}
    \setmainfont[]{Latin Modern Sans}
\fi
% use upquote if available, for straight quotes in verbatim environments
\IfFileExists{upquote.sty}{\usepackage{upquote}}{}
% use microtype if available
\IfFileExists{microtype.sty}{%
\usepackage{microtype}
\UseMicrotypeSet[protrusion]{basicmath} % disable protrusion for tt fonts
}{}
\usepackage{hyperref}
\hypersetup{unicode=true,
            pdftitle={Compte-rendu de travaux pratique no. 5},
            pdfborder={0 0 0},
            breaklinks=true}
\urlstyle{same}  % don't use monospace font for urls
\ifnum 0\ifxetex 1\fi\ifluatex 1\fi=0 % if pdftex
  \usepackage[shorthands=off,main=french]{babel}
\else
  \usepackage{polyglossia}
  \setmainlanguage[]{french}
\fi
\usepackage{color}
\usepackage{fancyvrb}
\newcommand{\VerbBar}{|}
\newcommand{\VERB}{\Verb[commandchars=\\\{\}]}
\DefineVerbatimEnvironment{Highlighting}{Verbatim}{commandchars=\\\{\}}
% Add ',fontsize=\small' for more characters per line
\newenvironment{Shaded}{}{}
\newcommand{\KeywordTok}[1]{\textcolor[rgb]{0.00,0.44,0.13}{\textbf{{#1}}}}
\newcommand{\DataTypeTok}[1]{\textcolor[rgb]{0.56,0.13,0.00}{{#1}}}
\newcommand{\DecValTok}[1]{\textcolor[rgb]{0.25,0.63,0.44}{{#1}}}
\newcommand{\BaseNTok}[1]{\textcolor[rgb]{0.25,0.63,0.44}{{#1}}}
\newcommand{\FloatTok}[1]{\textcolor[rgb]{0.25,0.63,0.44}{{#1}}}
\newcommand{\ConstantTok}[1]{\textcolor[rgb]{0.53,0.00,0.00}{{#1}}}
\newcommand{\CharTok}[1]{\textcolor[rgb]{0.25,0.44,0.63}{{#1}}}
\newcommand{\SpecialCharTok}[1]{\textcolor[rgb]{0.25,0.44,0.63}{{#1}}}
\newcommand{\StringTok}[1]{\textcolor[rgb]{0.25,0.44,0.63}{{#1}}}
\newcommand{\VerbatimStringTok}[1]{\textcolor[rgb]{0.25,0.44,0.63}{{#1}}}
\newcommand{\SpecialStringTok}[1]{\textcolor[rgb]{0.73,0.40,0.53}{{#1}}}
\newcommand{\ImportTok}[1]{{#1}}
\newcommand{\CommentTok}[1]{\textcolor[rgb]{0.38,0.63,0.69}{\textit{{#1}}}}
\newcommand{\DocumentationTok}[1]{\textcolor[rgb]{0.73,0.13,0.13}{\textit{{#1}}}}
\newcommand{\AnnotationTok}[1]{\textcolor[rgb]{0.38,0.63,0.69}{\textbf{\textit{{#1}}}}}
\newcommand{\CommentVarTok}[1]{\textcolor[rgb]{0.38,0.63,0.69}{\textbf{\textit{{#1}}}}}
\newcommand{\OtherTok}[1]{\textcolor[rgb]{0.00,0.44,0.13}{{#1}}}
\newcommand{\FunctionTok}[1]{\textcolor[rgb]{0.02,0.16,0.49}{{#1}}}
\newcommand{\VariableTok}[1]{\textcolor[rgb]{0.10,0.09,0.49}{{#1}}}
\newcommand{\ControlFlowTok}[1]{\textcolor[rgb]{0.00,0.44,0.13}{\textbf{{#1}}}}
\newcommand{\OperatorTok}[1]{\textcolor[rgb]{0.40,0.40,0.40}{{#1}}}
\newcommand{\BuiltInTok}[1]{{#1}}
\newcommand{\ExtensionTok}[1]{{#1}}
\newcommand{\PreprocessorTok}[1]{\textcolor[rgb]{0.74,0.48,0.00}{{#1}}}
\newcommand{\AttributeTok}[1]{\textcolor[rgb]{0.49,0.56,0.16}{{#1}}}
\newcommand{\RegionMarkerTok}[1]{{#1}}
\newcommand{\InformationTok}[1]{\textcolor[rgb]{0.38,0.63,0.69}{\textbf{\textit{{#1}}}}}
\newcommand{\WarningTok}[1]{\textcolor[rgb]{0.38,0.63,0.69}{\textbf{\textit{{#1}}}}}
\newcommand{\AlertTok}[1]{\textcolor[rgb]{1.00,0.00,0.00}{\textbf{{#1}}}}
\newcommand{\ErrorTok}[1]{\textcolor[rgb]{1.00,0.00,0.00}{\textbf{{#1}}}}
\newcommand{\NormalTok}[1]{{#1}}
\IfFileExists{parskip.sty}{%
\usepackage{parskip}
}{% else
\setlength{\parindent}{0pt}
\setlength{\parskip}{6pt plus 2pt minus 1pt}
}
\setlength{\emergencystretch}{3em}  % prevent overfull lines
\providecommand{\tightlist}{%
  \setlength{\itemsep}{0pt}\setlength{\parskip}{0pt}}
\setcounter{secnumdepth}{0}
% Redefines (sub)paragraphs to behave more like sections
\ifx\paragraph\undefined\else
\let\oldparagraph\paragraph
\renewcommand{\paragraph}[1]{\oldparagraph{#1}\mbox{}}
\fi
\ifx\subparagraph\undefined\else
\let\oldsubparagraph\subparagraph
\renewcommand{\subparagraph}[1]{\oldsubparagraph{#1}\mbox{}}
\fi

\title{Compte-rendu de travaux pratique no. 5}
\providecommand{\subtitle}[1]{}
\subtitle{Classification automatique de textures cycliques par analyse du plan de
Fourier}
\author{
Armand BOUR
 \and
Tristan CAMUS
}
\date{Jeudi 10 mars 2016}

\begin{document}
\maketitle

\section{Question 1}\label{question-1}

On sélectionne deux points capturant le motif répété afin d'obtenir la
période :
\[Période= |y(\text{Point 1}) - y(\text{Point 2})| = |19 - 24| = 5 \text{pixels / cycle}\]
\[Fréquence = \frac{1}{\text{période}} = \frac{1}{5} = 0.2 \text{cycle / pixel}\]

\section{Question 2}\label{question-2}

Les coordonnées de la raie maximale sont \((128, 128)\), soit
\((0, 0)\), soit l'origine, dans le plan de Fourier puisqu'il s'agit du
point central de l'image.

\section{Question 3}\label{question-3}

Les coordonnées de la raie secondaire peut se calculer en cherchant à
nouveau la valeur maximum à partir de l'image transformée où la raie
principale a été remplacée par un point noir. On obtient alors les
coordonnées \((128, 77)\), et \((128, 179)\), qui sont bien symétriques
par rapport au centre. Voici la macro modifiée afin de chercher les
raies secondaires :

\begin{Shaded}
\begin{Highlighting}[]
\NormalTok{macro }\StringTok{"direction FFT"}
\NormalTok{\{}
    \CommentTok{// ouverture d'une image si necessaire - sinon la macro analyse l'image courante}
    \CommentTok{//open ("/home/bmathon/Enseignement/TI/tp6_TF/images/256_a.jpg");}

    \CommentTok{// recuperation de l'identifiant de l'image}
    \NormalTok{image = }\FunctionTok{getImageID}\NormalTok{();}

    \CommentTok{// application de la TDF (FFT : Fast Fourier Transform)}
    \FunctionTok{run}\NormalTok{(}\StringTok{"FFT"}\NormalTok{);}

    \CommentTok{// recuperation de l'ID du module de la FFT}
    \NormalTok{fourier = }\FunctionTok{getImageID}\NormalTok{();}

    \CommentTok{// recuperation de la taille W x H du module de la FFT}
    \NormalTok{W = }\FunctionTok{getWidth}\NormalTok{();}
    \NormalTok{H = }\FunctionTok{getHeight}\NormalTok{();}

    \CommentTok{// recherche du max}
    \NormalTok{max_1 = }\DecValTok{0}\NormalTok{; }
    \NormalTok{i_max_1 = }\DecValTok{0}\NormalTok{;}
    \NormalTok{j_max_1 = }\DecValTok{0}\NormalTok{;}
    
    \KeywordTok{for} \NormalTok{(j=}\DecValTok{0}\NormalTok{; j<H; j++)}
    \NormalTok{\{}
        \KeywordTok{for} \NormalTok{(i=}\DecValTok{0}\NormalTok{; i<W; i++) }
        \NormalTok{\{}
            \NormalTok{p = }\FunctionTok{getPixel}\NormalTok{(i,j);}
            \KeywordTok{if} \NormalTok{( max_1 < p)}
            \NormalTok{\{}
                \NormalTok{max_1 =p;}
                \NormalTok{i_max_1 = i;}
                \NormalTok{j_max_1 =j;}
            \NormalTok{\} }
        \NormalTok{\}}
    \NormalTok{\}}

    \CommentTok{// mise a zero de la valeur max}

        \FunctionTok{setPixel} \NormalTok{(i_max_1,j_max_1,}\DecValTok{0}\NormalTok{);}
    \FunctionTok{print}\NormalTok{(}\StringTok{"i_max_1="}\NormalTok{, i_max_1);}
    \FunctionTok{print}\NormalTok{(}\StringTok{"j_max_1="}\NormalTok{, j_max_1);}


    \CommentTok{// Recherche de la raie secondaire en recherchant le max}
    \NormalTok{max_2 = }\DecValTok{0}\NormalTok{;}
    \NormalTok{i_max_2 = }\FunctionTok{newArray}\NormalTok{(}\DecValTok{2}\NormalTok{);}
    \NormalTok{j_max_2 = }\FunctionTok{newArray}\NormalTok{(}\DecValTok{2}\NormalTok{);}

    \KeywordTok{for} \NormalTok{(j = }\DecValTok{0}\NormalTok{; j < H; j++) \{}
        \KeywordTok{for} \NormalTok{(i = }\DecValTok{0}\NormalTok{; i < W; i++) \{}
            \NormalTok{p = }\FunctionTok{getPixel}\NormalTok{(i, j);}
            
            \KeywordTok{if} \NormalTok{(max_2 < p) \{}
                \NormalTok{max_2 = p;}
                \NormalTok{i_max_2[}\DecValTok{0}\NormalTok{] = i;}
                \NormalTok{j_max_2[}\DecValTok{0}\NormalTok{] = j;}
            \NormalTok{\} }\KeywordTok{else} \KeywordTok{if} \NormalTok{(max_2 == p) \{}
                \NormalTok{i_max_2[}\DecValTok{1}\NormalTok{] = i;}
                \NormalTok{j_max_2[}\DecValTok{1}\NormalTok{] = j;}
            \NormalTok{\}}
        \NormalTok{\}}
    \NormalTok{\}}

    \CommentTok{// En colore en noir les points correspondant aux raies secondaires}
    \FunctionTok{setPixel}\NormalTok{(i_max_2[}\DecValTok{0}\NormalTok{],  j_max_2[}\DecValTok{0}\NormalTok{], }\DecValTok{0}\NormalTok{);}
    \FunctionTok{setPixel}\NormalTok{(i_max_2[}\DecValTok{1}\NormalTok{],  j_max_2[}\DecValTok{1}\NormalTok{], }\DecValTok{0}\NormalTok{);}
    \CommentTok{// On affiche les coordonnées}
    \FunctionTok{print}\NormalTok{(}\StringTok{"i_max_2="}\NormalTok{, i_max_2[}\DecValTok{0}\NormalTok{], }\StringTok{" ; "}\NormalTok{, i_max_2[}\DecValTok{1}\NormalTok{]);}
    \FunctionTok{print}\NormalTok{(}\StringTok{"j_max_2="}\NormalTok{, j_max_2[}\DecValTok{0}\NormalTok{], }\StringTok{" ; "}\NormalTok{, j_max_2[}\DecValTok{1}\NormalTok{]);}
\NormalTok{\}}
\end{Highlighting}
\end{Shaded}

Afin de retrouver les coordonnées sur le plan de Fourier, il convient de
soustraire la moitié de la largeur de l'image à l'abscisse et de
soustraire l'ordonnée à la moitié de la hauteur de l'image dans le but
de replacer l'origine au centre de l'image. Ensuite, il ne reste plus
qu'à calculer le pourcentage de la coordonnée, et le réappliquer à la
différence entre la valeur maximum et la valeur minimum de l'axe.
Concrètement :
\[abscisse(y) = \frac{o(y) - \text{moitié de la largeur} } {\text{largeur}} \times (\text{valeur maximum de l’axe d’arrivée} - \text{valeur minimum de l’axe d’arrivée})\]
\[ordonnée(y) = \frac{\text{moitié de la hauteur} - o(y)} {\text{hauteur}} \times (\text{valeur maximum de l’axe d’arrivée} - \text{valeur minimum de l’axe d’arrivée})\]
Soit pour notre cas :
\[ordonnée(77) = \frac{\frac{256} {2} - 77} {256} \times (0.5 - (-0.5)) = \frac{51} {256} \approx 0.199\]
\[ordonnée(179) = \frac{\frac{256} {2} - 179} {256} \times (0.5 - (-0.5)) = -\frac{51} {256} \approx -0.199\]

Donc les coordonnées des raies secondaires sur le plan de Fourier sont :
\((0, 0.199)\) et \((0, -0.199)\).

\section{Question 4}\label{question-4}

\subsection{128\_a.jpg}\label{a.jpg}

La période est de 5 pixels / cycle. La fréquence est de 0,2
cycle/pixels. La raie principale se situe aux coordonnées \((64, 64)\),
soit \((0, 0)\) sur le plan de Fourier. Les raies secondaires se situent
aux cordonnées \((64, 38)\) et \((64, 90)\), soit \((0, 0.203)\) et
\((0, -0.203)\) sur le plan de Fourier.

\subsection{256\_b.jpg}\label{b.jpg}

La période est de 5 pixels / cycle. La fréquence est de 0,2
cycle/pixels. La raie principale se situe aux coordonnées
\((128, 128)\), soit \((0, 0)\) sur le plan de Fourier. Les raies
secondaires se situent aux cordonnées \((77, 77)\) et \((179, 179)\),
soit \((-0.199, 0.199)\) et \((0.199, -0.199)\) sur le plan de Fourier.

\section{Question 5}\label{question-5}

Si les abscisses des raies secondaires ont même signe, la texture est
\textbf{horizontale}. Si leurs ordonnées ont même signe, la texture est
\textbf{verticale}. Si les deux ont des signes différents, la texture
est \textbf{diagonale}. Ce qui donne en code :
\texttt{java\ \ \ \ \ if\ (i\_four{[}0{]}\ *\ i\_four{[}1{]}\ \textgreater{}=\ 0)\ \{\ \ \ \ \ \ \ \ \ class\ =\ "horizontale";\ \ \ \ \ \}\ else\ if\ (j\_four{[}0{]}\ *\ j\_four{[}1{]}\ \textgreater{}=\ 0)\ \{\ \ \ \ \ \ \ \ \ class\ =\ "verticale";\ \ \ \ \ \}\ else\ \{\ \ \ \ \ \ \ \ \ class\ =\ "diagonale";\ \ \ \ \ \}}

\end{document}
